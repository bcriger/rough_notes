% ##StateInjection ##MagicStates ##SurfaceCode
\documentclass[a4paper, english]{scrartcl}
\usepackage[utf8]{inputenc}
\usepackage{amsmath}
\usepackage{amsfonts}
\usepackage{amssymb}
\usepackage{babel}
\usepackage[cm]{fullpage}
\usepackage{float}
\usepackage{graphicx}
\usepackage{helvet}
\usepackage{hyperref}
\usepackage{mathtools}
\usepackage{nicefrac}
\usepackage{tikz}
\usepackage{pgfplots}
\usepackage{placeins}
\usepackage{verbatim}
\usepackage{xcolor}
\definecolor{gr}{gray}{0.9}
\renewcommand{\familydefault}{\sfdefault}
\title{State Injection}
\subtitle{Injecting into Rotated Surface Codes from the Corner}
\author{Ben Criger}
\date{\today}
\input{../../../Qcircuit.tex}
\usepackage{amsmath}
\usepackage{bbold}
\usepackage{color}
\usepackage{stmaryrd}
\usepackage{calc}
\usepackage{verbatim}
\usepackage{mathtools}
\usepackage{xspace}
\DeclarePairedDelimiter{\ceil}{\lceil}{\rceil}
\DeclarePairedDelimiter{\floor}{\lfloor}{\rfloor}
\DeclareMathOperator{\Span}{span}
\usepackage{tikz}
\usetikzlibrary{calc}
\providecommand{\polygon}[2]{%
  let \n{len} = {2*#2*tan(360/(2*#1))} in
 ++(0,-#2) ++(\n{len}/2,0) \foreach \x in {1,...,#1} { -- ++(\x*360/#1:\n{len})}}

\DeclareMathOperator\erf{erf}
\DeclareMathOperator\erfc{erfc}

\newsavebox\CBox
\newcommand\hcancel[2][0.5pt]{%
  \ifmmode\sbox\CBox{$#2$}\else\sbox\CBox{#2}\fi%
  \makebox[0pt][l]{\usebox\CBox}%  
  \rule[0.5\ht\CBox-#1/2]{\wd\CBox}{#1}}

\providecommand{\drv}[1]{\frac{\partial }{\partial #1}}
\providecommand{\drf}[2]{\frac{\partial #1}{\partial #2}}
\providecommand{\ddrf}[3]{\frac{\partial^2 #1}{\partial #2 \partial #3}}
\providecommand{\ddid}[3]{\frac{\partial^2 #1}{\partial #2 \partial #3} = \dfrac{\partial^2 #1}{\partial #3 \partial #2}}

\providecommand{\tr}{\mathrm{tr}}
 
\providecommand{\ket}[1]{\left \vert #1 \right \rangle}
\providecommand{\bra}[1]{\left \langle #1 \right \vert}
\providecommand{\braket}[2]{\left \langle #1 \left \vert #2 \right. \right \rangle}
\providecommand{\angles}[1]{\left \langle #1 \right \rangle}
\providecommand{\elem}[3]{\left \langle #1 \left \vert \vphantom{#1#2#3} #2 \right \vert #3 \right \rangle}
\providecommand{\delem}[2]{\left \langle #1 \left \vert \vphantom{#1#2} #2 \right \vert #1 \right \rangle}
\providecommand{\ketbra}[2]{\ket{#1} \! \bra{#2}}
\providecommand{\proj}[1]{\ketbra{#1}{#1}}
\providecommand{\twonorm}[1]{\| #1 \|_2}
\providecommand{\abs}[1]{\left \vert #1 \right \vert}
\providecommand{\set}[1]{\left \lbrace #1 \right \rbrace}
\providecommand{\group}[1]{\left \langle #1 \right \rangle}
\providecommand{\red}[1]{\textcolor[rgb]{0.5,0,0}{#1}}
\providecommand{\blue}[1]{\textcolor[rgb]{0,0,0.5}{#1}}
\providecommand{\green}[1]{\textcolor[rgb]{0,0.5,0}{#1}}
\providecommand{\conjecture}[1]{\red{#1 (check this).}}
\providecommand{\future}[1]{\green{#1 (do this later).}}
\providecommand{\id}{\hat{\mathbb{1}}}
\providecommand{\com}[2]{\left[#1,\,#2 \right]}
\providecommand{\acom}[2]{\left \lbrace #1,\,#2 \right \rbrace}
\providecommand{\diss}[2]{\mathcal{D}\left[ #1 \right]\left( #2 \right)}
\providecommand{\meas}[2]{\mathcal{M}\left[ #1 \right]\left( #2 \right)}
\providecommand{\lindtwo}[2]{ #1 #2 #1^{\dagger} - \dfrac{1}{2} \left \lbrace #1^{\dagger} #1,\,#2 \right \rbrace }
\providecommand{\lindthree}[3]{ #1 #2 #3 - \dfrac{1}{2} \acom{#3 #1}{#2} }
\providecommand{\lindfour}[4]{ #1 #2 #3 - \dfrac{1}{2} \acom{#4}{#2} }
\providecommand{\meastwo}[2]{ #1 #2 + #2 #1^{\dagger} - \tr \left( #1 #2 + #2 #1^{\dagger} \right) #2 }
\providecommand{\tenscom}[4]{\com{#1\otimes #2}{#3 \otimes #4}=\dfrac{1}{2}\left( \com{#1}{#3} \otimes \acom{#2}{#4} + \acom{#1}{#3} \otimes \com{#2}{#4} \right)}
\providecommand{\tenscomsimple}[4]{\com{#1\otimes #2}{#3 \otimes #4} = #1 #3 \otimes \com{#2}{#4} + \com{#1}{#3} \otimes  #4 #2}
\providecommand{\tensacom}[4]{\acom{#1\otimes #2}{#3 \otimes #4}=\dfrac{1}{2}\left( \com{#1}{#3} \otimes \com{#2}{#4} + \acom{#1}{#3} \otimes \acom{#2}{#4} \right)}
\providecommand{\trace}[1]{\mathrm{tr} \left( #1 \right)}
\providecommand{\comp}{\mathop{\bigcirc}}
\providecommand{\swap}{\textsc{swap}\xspace}
\providecommand{\cnot}{\textsc{cnot}\xspace}
\providecommand{\cz}{\textsc{cz}\xspace}
\providecommand{\col}{\textrm{col}}
\providecommand{\qec}[3]{\llbracket #1,\,#2,\,#3 \rrbracket}
\providecommand{\set}[1]{\left \lbrace #1 \right \rbrace}
\renewcommand{\Im}{\textrm{Im}}
\renewcommand{\Re}{\textrm{Re}}
\providecommand{\lind}{\mathcal{L}}
%\providecommand{\norm}[2]{\left \vert \left \vert #2 \right \vert \right \vert_{#1}}
\providecommand{\supp}[1]{\mathrm{supp}\left( #1 \right)}
\providecommand{\given}{\, \middle \vert \,}
\providecommand{\suchthat}{\, \middle \vert \,}
\providecommand{\diag}[1]{\mathrm{diag}\left( #1 \right)}
\providecommand{\ct}{^{\dagger}}

\providecommand{\norm}[1]{\left \Vert #1 \right \Vert} %only works with amsmath

\providecommand{\ncrit}{$n_{\textrm{crit}}$\xspace}

\renewcommand{\th}{\textrm{th}}
\newcommand{\prob}[1]{p\left(#1 \right)}
\newcommand{\cprob}[2]{p\left(#1\,\left\vert\vphantom{#1#2}\right. #2\right)}

\makeatletter
\providecommand{\pr}[2]{p\left(#1\,\middle|\,#2\right)}
\newcommand{\pushright}[1]{\ifmeasuring@#1\else\omit\hfill$\displaystyle#1$\fi\ignorespaces}
\newcommand{\pushleft}[1]{\ifmeasuring@#1\else\omit$\displaystyle#1$\hfill\fi\ignorespaces}
\makeatother
\newlength\figureheight
\newlength\figurewidth
\setlength\figureheight{7cm}
\setlength\figurewidth{12cm}

\providecommand{\cnot}{\textsc{cnot}}

\usetikzlibrary{decorations.pathreplacing,decorations.pathmorphing}

\begin{document}
\maketitle
\section{Introduction}
I'm interested to replicate (or at least understand) the work of Ying Li in high-fidelity state injection in surface codes.
Namely, I'd like to realise it in rotated surface codes, which use fewer data qubits. 
This document details an attempt to do so, which may or may not pan out. 
The key detail is that Ying Li puts the magic state on the corner, and so avoids having lots of random stabilisers nearby. 
I'd like to see if we can pull off the same trick. 
\section{$ZX$-Calculus Snippet}
We can express state distillation as a type of state transfer or teleportation (which are equivalent in the $ZX$-calculus):
\begin{figure}[!ht]
\centering
\includegraphics[width=0.2\textwidth]{zx_state_injection.pdf}
\caption{A variation on the state transfer protocol from section 3.3.2 of Coecke/Duncan. 
We begin with an arbitrary state $\ket{\psi}$ and a $\overline{\ket{+}}$, then merge by projecting into the $+1$-eigenspace of $Z \otimes \bar{Z}$. 
The resulting state is $\overline{\ket{\psi}}$.}
\end{figure}

The task now is to find a code for which there's a weight-3 $\bar{Z}$ that goes around a corner. 
Here's one:
\begin{figure}[!ht]
\centering
\includegraphics[width=0.3\textwidth]{corner_code.pdf}
\caption{A code with the right $\bar{Z}$ in the right place.}
\end{figure}

The $\bar{X}$ operators are all weight-two, which lets us break the $\overline{\ket{+}}$ into two Bell pairs and a 4-CAT:
\begin{figure}[!ht]
\centering
\includegraphics[width=0.3\textwidth]{injection_initial_state.pdf}
\caption{The initial and final stabilisers for the deformation I'm going to try.}
\end{figure}
\FloatBarrier
In case it's unclear, here's the deformation from initial to final stabilisers/logicals:
\begin{flalign}
S_{\textrm{init}} &= \left \langle
X_{12} ,\, X_{36} ,\, X_{45} ,\, X_{47} ,\, X_{58}
\right. \nonumber \\
& \phantom{=} \left. Z_{12} ,\, Z_{36} ,\, Z_{4578} \right \rangle \nonumber \\ 
L_{\textrm{init}} &= \group{X_0 ,\, Z_0}
\end{flalign}
I'm going to label the new stabilisers/logicals by which new operator I measure:
\begin{flalign}
S_{Z_{0134}} &= \left \langle
X_{1245} ,\, X_{3467} ,\, X_{58} ,\, X_{3456}
\right. \nonumber \\
& \phantom{=} \left. Z_{12} ,\, Z_{36} ,\, Z_{4578} ,\, Z_{0134} \right \rangle \nonumber \\ 
L_{Z_{0134}} &= \group{X_{012} ,\, Z_0} \\ %%%%%%%%%%
S_{Z_{67}} &= \left \langle
X_{1245} ,\, X_{3467} ,\, X_{58}
\right. \nonumber \\
& \phantom{=} \left. Z_{12} ,\, Z_{36} ,\, Z_{4578} ,\, Z_{0134} ,\, Z_{67} \right \rangle \nonumber \\ 
L_{Z_{67}} &= \group{X_{012} ,\, Z_0} \\ %%%%%%%%%%
S_{X_{03}} &= \left \langle
X_{1245} ,\, X_{3467} ,\, X_{58} ,\, X_{03}
\right. \nonumber \\
& \phantom{=} \left. Z_{12} ,\, Z_{4578} ,\, Z_{0134} ,\, Z_{67} \right \rangle \nonumber \\ 
L_{X_{03}} &= \group{X_{012} ,\, Z_{036}} %%%%%%%%%%
\end{flalign}
\section{Open Questions}
\begin{itemize}
\item Can this be made fault-tolerant?
\item What happens if we express state transfer as a bigger merge between multiple codes? 
\item How about we express a unitary encoder for the surface code in the $ZX$-calculus and translate it into a series of measurements?
\item How do we order and repeat measurements to post-select?
\end{itemize}
\end{document} 