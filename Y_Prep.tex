% ##QuantumErrorCorrection ##LatticeSurgery ##StatePreparation
\documentclass[a4paper, english]{scrartcl}
\usepackage[utf8]{inputenc}
\usepackage{amsmath}
\usepackage{amsfonts}
\usepackage{amssymb}
\usepackage{babel}
\usepackage[cm]{fullpage}
\usepackage{float}
\usepackage{graphicx}
\usepackage{helvet}
\usepackage{hyperref}
\usepackage{mathtools}
\usepackage{nicefrac}
\usepackage{tikz}
\usepackage{pgfplots}
\usepackage{placeins}
\usepackage{verbatim}
\usepackage{xcolor}
\definecolor{gr}{gray}{0.9}
\renewcommand{\familydefault}{\sfdefault}
\title{Quantum Fault Tolerance}
\subtitle{$Y$-Eigenstate Preparation}
\author{Ben Criger}
\date{\today}
\input{Qcircuit.tex}
\usepackage{amsmath}
\usepackage{bbold}
\usepackage{color}
\usepackage{stmaryrd}
\usepackage{calc}
\usepackage{verbatim}
\usepackage{mathtools}
\usepackage{xspace}
\DeclarePairedDelimiter{\ceil}{\lceil}{\rceil}
\DeclarePairedDelimiter{\floor}{\lfloor}{\rfloor}
\DeclareMathOperator{\Span}{span}
\usepackage{tikz}
\usetikzlibrary{calc}
\providecommand{\polygon}[2]{%
  let \n{len} = {2*#2*tan(360/(2*#1))} in
 ++(0,-#2) ++(\n{len}/2,0) \foreach \x in {1,...,#1} { -- ++(\x*360/#1:\n{len})}}

\DeclareMathOperator\erf{erf}
\DeclareMathOperator\erfc{erfc}

\newsavebox\CBox
\newcommand\hcancel[2][0.5pt]{%
  \ifmmode\sbox\CBox{$#2$}\else\sbox\CBox{#2}\fi%
  \makebox[0pt][l]{\usebox\CBox}%  
  \rule[0.5\ht\CBox-#1/2]{\wd\CBox}{#1}}

\providecommand{\drv}[1]{\frac{\partial }{\partial #1}}
\providecommand{\drf}[2]{\frac{\partial #1}{\partial #2}}
\providecommand{\ddrf}[3]{\frac{\partial^2 #1}{\partial #2 \partial #3}}
\providecommand{\ddid}[3]{\frac{\partial^2 #1}{\partial #2 \partial #3} = \dfrac{\partial^2 #1}{\partial #3 \partial #2}}

\providecommand{\tr}{\mathrm{tr}}
 
\providecommand{\ket}[1]{\left \vert #1 \right \rangle}
\providecommand{\bra}[1]{\left \langle #1 \right \vert}
\providecommand{\braket}[2]{\left \langle #1 \left \vert #2 \right. \right \rangle}
\providecommand{\angles}[1]{\left \langle #1 \right \rangle}
\providecommand{\elem}[3]{\left \langle #1 \left \vert \vphantom{#1#2#3} #2 \right \vert #3 \right \rangle}
\providecommand{\delem}[2]{\left \langle #1 \left \vert \vphantom{#1#2} #2 \right \vert #1 \right \rangle}
\providecommand{\ketbra}[2]{\ket{#1} \! \bra{#2}}
\providecommand{\proj}[1]{\ketbra{#1}{#1}}
\providecommand{\twonorm}[1]{\| #1 \|_2}
\providecommand{\abs}[1]{\left \vert #1 \right \vert}
\providecommand{\set}[1]{\left \lbrace #1 \right \rbrace}
\providecommand{\group}[1]{\left \langle #1 \right \rangle}
\providecommand{\red}[1]{\textcolor[rgb]{0.5,0,0}{#1}}
\providecommand{\blue}[1]{\textcolor[rgb]{0,0,0.5}{#1}}
\providecommand{\green}[1]{\textcolor[rgb]{0,0.5,0}{#1}}
\providecommand{\conjecture}[1]{\red{#1 (check this).}}
\providecommand{\future}[1]{\green{#1 (do this later).}}
\providecommand{\id}{\hat{\mathbb{1}}}
\providecommand{\com}[2]{\left[#1,\,#2 \right]}
\providecommand{\acom}[2]{\left \lbrace #1,\,#2 \right \rbrace}
\providecommand{\diss}[2]{\mathcal{D}\left[ #1 \right]\left( #2 \right)}
\providecommand{\meas}[2]{\mathcal{M}\left[ #1 \right]\left( #2 \right)}
\providecommand{\lindtwo}[2]{ #1 #2 #1^{\dagger} - \dfrac{1}{2} \left \lbrace #1^{\dagger} #1,\,#2 \right \rbrace }
\providecommand{\lindthree}[3]{ #1 #2 #3 - \dfrac{1}{2} \acom{#3 #1}{#2} }
\providecommand{\lindfour}[4]{ #1 #2 #3 - \dfrac{1}{2} \acom{#4}{#2} }
\providecommand{\meastwo}[2]{ #1 #2 + #2 #1^{\dagger} - \tr \left( #1 #2 + #2 #1^{\dagger} \right) #2 }
\providecommand{\tenscom}[4]{\com{#1\otimes #2}{#3 \otimes #4}=\dfrac{1}{2}\left( \com{#1}{#3} \otimes \acom{#2}{#4} + \acom{#1}{#3} \otimes \com{#2}{#4} \right)}
\providecommand{\tenscomsimple}[4]{\com{#1\otimes #2}{#3 \otimes #4} = #1 #3 \otimes \com{#2}{#4} + \com{#1}{#3} \otimes  #4 #2}
\providecommand{\tensacom}[4]{\acom{#1\otimes #2}{#3 \otimes #4}=\dfrac{1}{2}\left( \com{#1}{#3} \otimes \com{#2}{#4} + \acom{#1}{#3} \otimes \acom{#2}{#4} \right)}
\providecommand{\trace}[1]{\mathrm{tr} \left( #1 \right)}
\providecommand{\comp}{\mathop{\bigcirc}}
\providecommand{\swap}{\textsc{swap}\xspace}
\providecommand{\cnot}{\textsc{cnot}\xspace}
\providecommand{\cz}{\textsc{cz}\xspace}
\providecommand{\col}{\textrm{col}}
\providecommand{\qec}[3]{\llbracket #1,\,#2,\,#3 \rrbracket}
\providecommand{\set}[1]{\left \lbrace #1 \right \rbrace}
\renewcommand{\Im}{\textrm{Im}}
\renewcommand{\Re}{\textrm{Re}}
\providecommand{\lind}{\mathcal{L}}
%\providecommand{\norm}[2]{\left \vert \left \vert #2 \right \vert \right \vert_{#1}}
\providecommand{\supp}[1]{\mathrm{supp}\left( #1 \right)}
\providecommand{\given}{\, \middle \vert \,}
\providecommand{\suchthat}{\, \middle \vert \,}
\providecommand{\diag}[1]{\mathrm{diag}\left( #1 \right)}
\providecommand{\ct}{^{\dagger}}

\providecommand{\norm}[1]{\left \Vert #1 \right \Vert} %only works with amsmath

\providecommand{\ncrit}{$n_{\textrm{crit}}$\xspace}

\renewcommand{\th}{\textrm{th}}
\newcommand{\prob}[1]{p\left(#1 \right)}
\newcommand{\cprob}[2]{p\left(#1\,\left\vert\vphantom{#1#2}\right. #2\right)}

\makeatletter
\providecommand{\pr}[2]{p\left(#1\,\middle|\,#2\right)}
\newcommand{\pushright}[1]{\ifmeasuring@#1\else\omit\hfill$\displaystyle#1$\fi\ignorespaces}
\newcommand{\pushleft}[1]{\ifmeasuring@#1\else\omit$\displaystyle#1$\hfill\fi\ignorespaces}
\makeatother
\newlength\figureheight
\newlength\figurewidth
\setlength\figureheight{7cm}
\setlength\figurewidth{12cm}

\providecommand{\cnot}{\textsc{cnot}}

\begin{document}
\maketitle
\section{Introduction}
People always act like fault-tolerant state preparation by measurement is limited to the $X$ or $Z$ eigenstates. 
This implies that, in order to prepare a $Y$ eigenstate, we have to use state distillation. 
Let's confirm that this is the case by attempting some na\"{i}ve measurement-based preparation and showing that it doesn't work. 
\section{Surface Code}
We begin with nine qubits in $Y$ eigenstates.
The checks to be measured are:
\begin{equation}
X_{01},\,\,X_{1245},\,\,X_{3467},\,\,X_{78},\,\,Z_{25},\,\,Z_{4578},\,\,Z_{0134},\,\,Z_{36}
\end{equation}
We measure in the $X$/$Z$ boundary operators first:
\begin{flalign}
S &\mapsto \group{Y_4,\,(-1)^{x_{01}}X_{01}, \, Y_{01}
,\,(-1)^{x_{78}}X_{78}, \, Y_{78}
,\,(-1)^{z_{25}}Z_{25}, \, Y_{25}
,\,(-1)^{z_{36}}Z_{36}, \, Y_{36}} \\
&\underset{M_{X_{1245}}}{\mapsto} \group{(-1)^{x_{1245}}X_{1245},\, Y_{014},\,(-1)^{x_{01}}X_{01}, \, 
(-1)^{x_{78}}X_{78}, \, Y_{78}
,\,(-1)^{z_{25}}Z_{25}, \, Y_{25}
,\,(-1)^{z_{36}}Z_{36}, \, Y_{36}} \\
&\underset{M_{X_{3467}}}{\mapsto} \left \langle (-1)^{x_{1245}}X_{1245},\, (-1)^{x_{3467}}X_{3467},\, (-1)^{x_{01}}X_{01},\, 
(-1)^{x_{78}}X_{78}
,\,(-1)^{z_{25}}Z_{25},\,(-1)^{z_{36}}Z_{36}, \right. \nonumber \\
&\qquad \qquad \left. Y_{01478}, \, Y_{25}
,\, Y_{36} \right \rangle \\
& \underset{M_{Z_{0134}}}{\mapsto} \left \langle (-1)^{x_{1245}}X_{1245},\,
(-1)^{z_{0134}}Z_{0134},\,
(-1)^{x_{3467}}X_{3467},\, (-1)^{x_{01}}X_{01},\, 
(-1)^{x_{78}}X_{78}
,\,(-1)^{z_{25}}Z_{25},\,(-1)^{z_{36}}Z_{36}, \right. \nonumber \\
&\qquad \qquad \left. Y_{0134678}, \, Y_{25} \right \rangle
\end{flalign}
Under the last stabiliser measurement, this maps to a joint eigenstate of the surface code stabilisers and a transversal $Y$ operator. 
This transversal operator is a logical $Y$ for any odd-distance surface code, since it is the product of an odd number of $Z$ columns and an odd number of $X$ rows. 

To correct for random measurement results, we have to find an even-weight $X$/$Z$ Pauli for every $-1$ syndrome, so that it will commute with the transversal $Y$.
This is always possible, since the distance is odd, so one of the $X$/$Z$ boundaries will be an even distance away from any given syndrome.
The catch is when you have to also correct a data error, which we go over in the next  section.

\section{Data Errors}
When we prepare an $X$ or $Z$ eigenstate, one type of stabiliser gives completely random results. 
This is tolerated, since the type of data error that the randomized stabiliser would detect is the same as the prepared basis state used at the beginning. 
For example, if preparing an $X$ eigenstate, we don't care about $X$ errors that occur on the data, since we begin in a $\ket{+}$ state on every qubit.
$Z$ errors are faithfully detected by the non-randomised stabilisers, and we can correct them. 
This is not the case when preparing a $Y$ eigenstate, since both sets of stabilisers give random results. 
\section{Open Question}
There are relatives of the surface code that have $Y$/$Z$ or $Y$/$X$ stabilisers instead of $X$/$Z$.
Some of the old and new stabilisers will anti-commute if we measure one code at one round and another code the next. 
Are these codes still useful?
\end{document}