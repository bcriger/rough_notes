% ##FaultTolerantOperations ##MagicStates ##Teleportation
\documentclass[10pt,a4paper, english]{scrartcl}
\usepackage[utf8]{inputenc}
\usepackage{amsmath}
\usepackage{amsfonts}
\usepackage{amssymb}
\usepackage{babel}
\usepackage[cm]{fullpage}
\usepackage{float}
\usepackage{graphicx}
\usepackage{helvet}
\usepackage{hyperref}
\usepackage{mathtools}
\usepackage{nicefrac}
\usepackage{tikz}
\usepackage{pgfplots}
\usepackage{placeins}
\usepackage{verbatim}
\usepackage{xcolor}
\definecolor{gr}{gray}{0.9}
\renewcommand{\familydefault}{\sfdefault}
\title{Magic States for Multi-Qubit Operations}
\author{Ben Criger}
\date{\today}
\input{Qcircuit.tex}
\usepackage{amsmath}
\usepackage{bbold}
\usepackage{color}
\usepackage{stmaryrd}
\usepackage{calc}
\usepackage{verbatim}
\usepackage{mathtools}
\usepackage{xspace}
\DeclarePairedDelimiter{\ceil}{\lceil}{\rceil}
\DeclarePairedDelimiter{\floor}{\lfloor}{\rfloor}
\DeclareMathOperator{\Span}{span}
\usepackage{tikz}
\usetikzlibrary{calc}
\providecommand{\polygon}[2]{%
  let \n{len} = {2*#2*tan(360/(2*#1))} in
 ++(0,-#2) ++(\n{len}/2,0) \foreach \x in {1,...,#1} { -- ++(\x*360/#1:\n{len})}}

\DeclareMathOperator\erf{erf}
\DeclareMathOperator\erfc{erfc}

\newsavebox\CBox
\newcommand\hcancel[2][0.5pt]{%
  \ifmmode\sbox\CBox{$#2$}\else\sbox\CBox{#2}\fi%
  \makebox[0pt][l]{\usebox\CBox}%  
  \rule[0.5\ht\CBox-#1/2]{\wd\CBox}{#1}}

\providecommand{\drv}[1]{\frac{\partial }{\partial #1}}
\providecommand{\drf}[2]{\frac{\partial #1}{\partial #2}}
\providecommand{\ddrf}[3]{\frac{\partial^2 #1}{\partial #2 \partial #3}}
\providecommand{\ddid}[3]{\frac{\partial^2 #1}{\partial #2 \partial #3} = \dfrac{\partial^2 #1}{\partial #3 \partial #2}}

\providecommand{\tr}{\mathrm{tr}}
 
\providecommand{\ket}[1]{\left \vert #1 \right \rangle}
\providecommand{\bra}[1]{\left \langle #1 \right \vert}
\providecommand{\braket}[2]{\left \langle #1 \left \vert #2 \right. \right \rangle}
\providecommand{\angles}[1]{\left \langle #1 \right \rangle}
\providecommand{\elem}[3]{\left \langle #1 \left \vert \vphantom{#1#2#3} #2 \right \vert #3 \right \rangle}
\providecommand{\delem}[2]{\left \langle #1 \left \vert \vphantom{#1#2} #2 \right \vert #1 \right \rangle}
\providecommand{\ketbra}[2]{\ket{#1} \! \bra{#2}}
\providecommand{\proj}[1]{\ketbra{#1}{#1}}
\providecommand{\twonorm}[1]{\| #1 \|_2}
\providecommand{\abs}[1]{\left \vert #1 \right \vert}
\providecommand{\set}[1]{\left \lbrace #1 \right \rbrace}
\providecommand{\group}[1]{\left \langle #1 \right \rangle}
\providecommand{\red}[1]{\textcolor[rgb]{0.5,0,0}{#1}}
\providecommand{\blue}[1]{\textcolor[rgb]{0,0,0.5}{#1}}
\providecommand{\green}[1]{\textcolor[rgb]{0,0.5,0}{#1}}
\providecommand{\conjecture}[1]{\red{#1 (check this).}}
\providecommand{\future}[1]{\green{#1 (do this later).}}
\providecommand{\id}{\hat{\mathbb{1}}}
\providecommand{\com}[2]{\left[#1,\,#2 \right]}
\providecommand{\acom}[2]{\left \lbrace #1,\,#2 \right \rbrace}
\providecommand{\diss}[2]{\mathcal{D}\left[ #1 \right]\left( #2 \right)}
\providecommand{\meas}[2]{\mathcal{M}\left[ #1 \right]\left( #2 \right)}
\providecommand{\lindtwo}[2]{ #1 #2 #1^{\dagger} - \dfrac{1}{2} \left \lbrace #1^{\dagger} #1,\,#2 \right \rbrace }
\providecommand{\lindthree}[3]{ #1 #2 #3 - \dfrac{1}{2} \acom{#3 #1}{#2} }
\providecommand{\lindfour}[4]{ #1 #2 #3 - \dfrac{1}{2} \acom{#4}{#2} }
\providecommand{\meastwo}[2]{ #1 #2 + #2 #1^{\dagger} - \tr \left( #1 #2 + #2 #1^{\dagger} \right) #2 }
\providecommand{\tenscom}[4]{\com{#1\otimes #2}{#3 \otimes #4}=\dfrac{1}{2}\left( \com{#1}{#3} \otimes \acom{#2}{#4} + \acom{#1}{#3} \otimes \com{#2}{#4} \right)}
\providecommand{\tenscomsimple}[4]{\com{#1\otimes #2}{#3 \otimes #4} = #1 #3 \otimes \com{#2}{#4} + \com{#1}{#3} \otimes  #4 #2}
\providecommand{\tensacom}[4]{\acom{#1\otimes #2}{#3 \otimes #4}=\dfrac{1}{2}\left( \com{#1}{#3} \otimes \com{#2}{#4} + \acom{#1}{#3} \otimes \acom{#2}{#4} \right)}
\providecommand{\trace}[1]{\mathrm{tr} \left( #1 \right)}
\providecommand{\comp}{\mathop{\bigcirc}}
\providecommand{\swap}{\textsc{swap}\xspace}
\providecommand{\cnot}{\textsc{cnot}\xspace}
\providecommand{\cz}{\textsc{cz}\xspace}
\providecommand{\col}{\textrm{col}}
\providecommand{\qec}[3]{\llbracket #1,\,#2,\,#3 \rrbracket}
\providecommand{\set}[1]{\left \lbrace #1 \right \rbrace}
\renewcommand{\Im}{\textrm{Im}}
\renewcommand{\Re}{\textrm{Re}}
\providecommand{\lind}{\mathcal{L}}
%\providecommand{\norm}[2]{\left \vert \left \vert #2 \right \vert \right \vert_{#1}}
\providecommand{\supp}[1]{\mathrm{supp}\left( #1 \right)}
\providecommand{\given}{\, \middle \vert \,}
\providecommand{\suchthat}{\, \middle \vert \,}
\providecommand{\diag}[1]{\mathrm{diag}\left( #1 \right)}
\providecommand{\ct}{^{\dagger}}

\providecommand{\norm}[1]{\left \Vert #1 \right \Vert} %only works with amsmath

\providecommand{\ncrit}{$n_{\textrm{crit}}$\xspace}

\renewcommand{\th}{\textrm{th}}
\newcommand{\prob}[1]{p\left(#1 \right)}
\newcommand{\cprob}[2]{p\left(#1\,\left\vert\vphantom{#1#2}\right. #2\right)}

\makeatletter
\providecommand{\pr}[2]{p\left(#1\,\middle|\,#2\right)}
\newcommand{\pushright}[1]{\ifmeasuring@#1\else\omit\hfill$\displaystyle#1$\fi\ignorespaces}
\newcommand{\pushleft}[1]{\ifmeasuring@#1\else\omit$\displaystyle#1$\hfill\fi\ignorespaces}
\makeatother
\newlength\figureheight
\newlength\figurewidth
\setlength\figureheight{7cm}
\setlength\figurewidth{12cm}

\providecommand{\cnot}{\textsc{cnot}}

\usetikzlibrary{decorations.pathreplacing,decorations.pathmorphing}

\begin{document}
\maketitle
\section{Introduction}
The purpose of this note is to determine the utility of single-qubit magic states for performing multi-qubit operations.
All this may already be known, but I'm not looking it up, for whatever reason. 
I start with a review of single-qubit gate teleportation, then try to phrase this protocol as a joint measurement protocol.
I conclude (I hope) by generalising this protocol to multi-qubit measurements. 
\section{Review}
Here's the well-known protocol for implementing a $T$ gate using Clifford operations and state preparation:
\begin{figure}[!h]
\centering
\input{GateTeleportation.tikz}
\end{figure}
\FloatBarrier
To show that this circuit does what it's supposed to, we can do a little math:
\begin{flalign}
\ket{\psi_{\mathrm{init}}} &= \left( \ket{0} + \exp \left( \frac{i\pi}{4} \right) \ket{1} \right) \otimes \left( \alpha \ket{0} + \beta \ket{1} \right) = \alpha \ket{00} + \beta \ket{01} + \alpha \exp\left( \frac{i\pi}{4} \right) \ket{10} + \beta \exp\left( \frac{i\pi}{4} \right) \ket{11} \\
\mathrm{CNOT}\ket{\psi_{\mathrm{init}}} &= \alpha \ket{00} + \beta \ket{01} + \alpha \exp\left( \frac{i\pi}{4} \right) \ket{11} + \beta \exp\left( \frac{i\pi}{4} \right) \ket{10} \nonumber \\
&= \left(\alpha \ket{0} + \beta \exp \left( \frac{i\pi}{4} \right)  \right) \otimes \ket{0} + \left( \alpha \exp \left( \frac{i\pi}{4} \right) \ket{1} + \beta \ket{0}\right) \otimes \ket{1} \\
&= \left(\alpha \ket{0} + \beta \exp \left( \frac{i\pi}{4} \right)  \right) \otimes \ket{0} + S^{-1}X\left(\exp \left( \frac{-i\pi}{4} \right)\alpha \ket{0} + \beta \ket{1}\right) \otimes \ket{1}
\end{flalign}
If we measure the bottom qubit, obtaining $0$ or $1$, the post-selected states are equivalent up to a global phase after applying the classically-controlled $XS$. 

%Bad writing:
To try to generalise this protocol, it might be useful to phrase it in terms of joint Pauli measurements.
\section{Joint Measurement Protocol}
If we propagate the measured operator $Z_1$ back through the \cnot, we get a joint $Z_0Z_1$. 
Measuring this operator projects either into the space spanned by $\set{\ket{00},\, \ket{11}}$ or $\set{\ket{01},\, \ket{10}}$. 
At this point, we could do the classically-controlled Clifford before we do the \cnot, but it's tough to see the general principle at work. 

Though, maybe you could disentangle by measuring an $X$ or something.
\section{Multi-Qubit Measurement}
\section{Questions}
\begin{enumerate}
\item Taking a look at Daniel Gottesman's thesis, near equation 5.23, it looks like there's a state that can be prepared from post-selected joint Pauli measurement that facilitates a Toffoli gate. It would be interesting to see if this can be extended to other fixed-parity multi-qubit states, and what the resource count looks like. \textbf{Response:} don't you need some kind of multi-control $Z$ to get that state?
\item Can you get a $T$ gate from a Toffoli gate? I remember there being some conversion in Nielsen/Chuang, but it was between the Toffoli gate and some other fancy angle. 
\end{enumerate}

\end{document}