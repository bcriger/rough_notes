% ##OpenQuantumSystems
\documentclass[10pt,a4paper, english]{scrartcl}
\usepackage[utf8]{inputenc}
\usepackage{amsmath}
\usepackage{amsfonts}
\usepackage{amssymb}
\usepackage{babel}
\usepackage[cm]{fullpage}
\usepackage{float}
\usepackage{graphicx}
\usepackage{helvet}
\usepackage{hyperref}
\usepackage{mathtools}
\usepackage{nicefrac}
\usepackage{tikz}
\usepackage{pgfplots}
\usepackage{placeins}
\usepackage{verbatim}
\usepackage{xcolor}
\definecolor{gr}{gray}{0.9}
\renewcommand{\familydefault}{\sfdefault}
\title{Effect of Decoherence During Gates}
\author{Ben Criger}
\date{\today}
\input{Qcircuit.tex}
\usepackage{amsmath}
\usepackage{bbold}
\usepackage{color}
\usepackage{stmaryrd}
\usepackage{calc}
\usepackage{verbatim}
\usepackage{mathtools}
\usepackage{xspace}
\DeclarePairedDelimiter{\ceil}{\lceil}{\rceil}
\DeclarePairedDelimiter{\floor}{\lfloor}{\rfloor}
\DeclareMathOperator{\Span}{span}
\usepackage{tikz}
\usetikzlibrary{calc}
\providecommand{\polygon}[2]{%
  let \n{len} = {2*#2*tan(360/(2*#1))} in
 ++(0,-#2) ++(\n{len}/2,0) \foreach \x in {1,...,#1} { -- ++(\x*360/#1:\n{len})}}

\DeclareMathOperator\erf{erf}
\DeclareMathOperator\erfc{erfc}

\newsavebox\CBox
\newcommand\hcancel[2][0.5pt]{%
  \ifmmode\sbox\CBox{$#2$}\else\sbox\CBox{#2}\fi%
  \makebox[0pt][l]{\usebox\CBox}%  
  \rule[0.5\ht\CBox-#1/2]{\wd\CBox}{#1}}

\providecommand{\drv}[1]{\frac{\partial }{\partial #1}}
\providecommand{\drf}[2]{\frac{\partial #1}{\partial #2}}
\providecommand{\ddrf}[3]{\frac{\partial^2 #1}{\partial #2 \partial #3}}
\providecommand{\ddid}[3]{\frac{\partial^2 #1}{\partial #2 \partial #3} = \dfrac{\partial^2 #1}{\partial #3 \partial #2}}

\providecommand{\tr}{\mathrm{tr}}
 
\providecommand{\ket}[1]{\left \vert #1 \right \rangle}
\providecommand{\bra}[1]{\left \langle #1 \right \vert}
\providecommand{\braket}[2]{\left \langle #1 \left \vert #2 \right. \right \rangle}
\providecommand{\angles}[1]{\left \langle #1 \right \rangle}
\providecommand{\elem}[3]{\left \langle #1 \left \vert \vphantom{#1#2#3} #2 \right \vert #3 \right \rangle}
\providecommand{\delem}[2]{\left \langle #1 \left \vert \vphantom{#1#2} #2 \right \vert #1 \right \rangle}
\providecommand{\ketbra}[2]{\ket{#1} \! \bra{#2}}
\providecommand{\proj}[1]{\ketbra{#1}{#1}}
\providecommand{\twonorm}[1]{\| #1 \|_2}
\providecommand{\abs}[1]{\left \vert #1 \right \vert}
\providecommand{\set}[1]{\left \lbrace #1 \right \rbrace}
\providecommand{\group}[1]{\left \langle #1 \right \rangle}
\providecommand{\red}[1]{\textcolor[rgb]{0.5,0,0}{#1}}
\providecommand{\blue}[1]{\textcolor[rgb]{0,0,0.5}{#1}}
\providecommand{\green}[1]{\textcolor[rgb]{0,0.5,0}{#1}}
\providecommand{\conjecture}[1]{\red{#1 (check this).}}
\providecommand{\future}[1]{\green{#1 (do this later).}}
\providecommand{\id}{\hat{\mathbb{1}}}
\providecommand{\com}[2]{\left[#1,\,#2 \right]}
\providecommand{\acom}[2]{\left \lbrace #1,\,#2 \right \rbrace}
\providecommand{\diss}[2]{\mathcal{D}\left[ #1 \right]\left( #2 \right)}
\providecommand{\meas}[2]{\mathcal{M}\left[ #1 \right]\left( #2 \right)}
\providecommand{\lindtwo}[2]{ #1 #2 #1^{\dagger} - \dfrac{1}{2} \left \lbrace #1^{\dagger} #1,\,#2 \right \rbrace }
\providecommand{\lindthree}[3]{ #1 #2 #3 - \dfrac{1}{2} \acom{#3 #1}{#2} }
\providecommand{\lindfour}[4]{ #1 #2 #3 - \dfrac{1}{2} \acom{#4}{#2} }
\providecommand{\meastwo}[2]{ #1 #2 + #2 #1^{\dagger} - \tr \left( #1 #2 + #2 #1^{\dagger} \right) #2 }
\providecommand{\tenscom}[4]{\com{#1\otimes #2}{#3 \otimes #4}=\dfrac{1}{2}\left( \com{#1}{#3} \otimes \acom{#2}{#4} + \acom{#1}{#3} \otimes \com{#2}{#4} \right)}
\providecommand{\tenscomsimple}[4]{\com{#1\otimes #2}{#3 \otimes #4} = #1 #3 \otimes \com{#2}{#4} + \com{#1}{#3} \otimes  #4 #2}
\providecommand{\tensacom}[4]{\acom{#1\otimes #2}{#3 \otimes #4}=\dfrac{1}{2}\left( \com{#1}{#3} \otimes \com{#2}{#4} + \acom{#1}{#3} \otimes \acom{#2}{#4} \right)}
\providecommand{\trace}[1]{\mathrm{tr} \left( #1 \right)}
\providecommand{\comp}{\mathop{\bigcirc}}
\providecommand{\swap}{\textsc{swap}\xspace}
\providecommand{\cnot}{\textsc{cnot}\xspace}
\providecommand{\cz}{\textsc{cz}\xspace}
\providecommand{\col}{\textrm{col}}
\providecommand{\qec}[3]{\llbracket #1,\,#2,\,#3 \rrbracket}
\providecommand{\set}[1]{\left \lbrace #1 \right \rbrace}
\renewcommand{\Im}{\textrm{Im}}
\renewcommand{\Re}{\textrm{Re}}
\providecommand{\lind}{\mathcal{L}}
%\providecommand{\norm}[2]{\left \vert \left \vert #2 \right \vert \right \vert_{#1}}
\providecommand{\supp}[1]{\mathrm{supp}\left( #1 \right)}
\providecommand{\given}{\, \middle \vert \,}
\providecommand{\suchthat}{\, \middle \vert \,}
\providecommand{\diag}[1]{\mathrm{diag}\left( #1 \right)}
\providecommand{\ct}{^{\dagger}}

\providecommand{\norm}[1]{\left \Vert #1 \right \Vert} %only works with amsmath

\providecommand{\ncrit}{$n_{\textrm{crit}}$\xspace}

\renewcommand{\th}{\textrm{th}}
\newcommand{\prob}[1]{p\left(#1 \right)}
\newcommand{\cprob}[2]{p\left(#1\,\left\vert\vphantom{#1#2}\right. #2\right)}

\makeatletter
\providecommand{\pr}[2]{p\left(#1\,\middle|\,#2\right)}
\newcommand{\pushright}[1]{\ifmeasuring@#1\else\omit\hfill$\displaystyle#1$\fi\ignorespaces}
\newcommand{\pushleft}[1]{\ifmeasuring@#1\else\omit$\displaystyle#1$\hfill\fi\ignorespaces}
\makeatother
\newlength\figureheight
\newlength\figurewidth
\setlength\figureheight{7cm}
\setlength\figurewidth{12cm}

\providecommand{\cnot}{\textsc{cnot}}

\usetikzlibrary{decorations.pathreplacing,decorations.pathmorphing}

\begin{document}
\maketitle
\section{Introduction}
The purpose of this note is to determine the effect of biased (read: mostly dephasing) noise during the operation of certain gates.
I'm going to start with the $Y_{90}$ gate, implemented by a nice square pulse. 
There are two limits that can be handled succinctly, the perfectly-Markovian limit (where we can't decrease the error rate without a quantum error-correcting code), and the `unknown constant Hamiltonian' limit, where we can apply pulse sequences, reversing the effect of the unknown Hamiltonian for $\sim \nicefrac{1}{2}$ the pulse duration.
I begin with the Markovian limit, as I'm not so familiar with DD pulse sequences.
\section{Helpful Math}
Let's vectorize the density matrix, in the Pauli basis:
\begin{equation}
\rho = \frac{\id}{2} + \rho_x \sigma_x + \rho_y \sigma_y + \rho_z \sigma_z 
\end{equation}
To express the equation of motion in this basis, I calculate a few commutators and dissipators. 
\begin{flalign}
&\diss{A}{B} = \lindtwo{A}{B} \\
% &\diss{\id}{B} = \lindtwo{\id}{B} = 0 \\
&\diss{A}{\id} = \lindtwo{A}{\id} = \com{A}{A\ct}\\
&\therefore 
% &\diss{\sigma_+}{\id} = \com{\sigma_+}{\sigma_-} = -\sigma_z \quad \diss{\sigma_-}{\id} = \com{\sigma_-}{\sigma_+} = \sigma_z \quad
\diss{\sigma_z}{\id} = \com{\sigma_z}{\sigma_z} = 0 \\
&\diss{\sigma_z}{\sigma_z} = 0\\
&\diss{\sigma_z}{\sigma_{x,y}} = -2\sigma_{x,y}\\
% & \diss{\sigma_-}{\sigma_{x,y,z}} = \lindthree{\ketbra{0}{1}}{\sigma_{x,y,z}}{\ketbra{1}{0}}\\
% & \diss{\sigma_-}{\sigma_{x,y}} = -\dfrac{1}{2}\sigma_{x,y}\\
% &\diss{\sigma_-}{\sigma_z} = -\sigma_z \\ 
&\com{\sigma_y}{\sigma_x} = -2i\sigma_z \quad \com{\sigma_y}{\sigma_z} = 2i\sigma_x 
\end{flalign}
With these in hand, we can start expressing the equation of motion for a noisy $Y_{90}$ in this operator basis. 
\section{$Y_{90}$ with Markovian Noise}
A simple master equation for a $Y_{90}$, subject to dephasing is:
\begin{equation}
\dot{\rho} = -i\frac{\omega}{2} \com{\sigma_y}{\rho} + \frac{\gamma}{2} \diss{\sigma_z}{\rho},
\end{equation}
where the factors of two are included to make the matrix description look nice, as we will see in a minute. 
I express the commutator and dissipator in matrix form:
\begin{alignat}{2}
\frac{\gamma}{2} \diss{\sigma_z}{\cdot} &= \begin{bmatrix}
0 & 0 & 0 & 0 \\ 0 & -\gamma & 0 & 0 \\ 0 & 0 & -\gamma & 0 \\ 0 & 0 & 0 & 0
\end{bmatrix}
\quad 
-i \frac{\omega}{2} \com{\sigma_y}{\cdot} &&= \begin{bmatrix}
0 & 0 & 0 & 0 \\ 0 & 0 & 0 & \omega \\ 0 & 0 & 0 & 0 \\ 0 & -\omega & 0 & 0
\end{bmatrix}.
\end{alignat}
The Lindbladian is just the sum of these two terms:
\begin{equation}
\dot{\vec{\rho}} = \begin{bmatrix}
0 & 0 & 0 & 0 \\ 0 & -\gamma & 0 & \omega \\ 0 & 0 & -\gamma & 0 \\ 0 & -\omega & 0 & 0
\end{bmatrix} \vec{\rho} = \hat{L} \vec{\rho}
\end{equation}
We take the matrix exponent $\exp(\hat{L} t)$ to get the superoperator $S$:
\begin{equation}
S = \begin{bmatrix}
1 & 0 & 0 & 0 \\
0 & e^{-\frac{\gamma t}{2}}\left( \cosh(\nicefrac{\beta t}{2}) - \frac{\gamma}{\beta} \sinh(\nicefrac{\beta t}{2})\right) & 0 & -\frac{2 \omega}{\beta} e^{-\frac{\gamma t}{2}}\sinh(\nicefrac{\beta t}{2}) \\
0 & 0 & e^{-\frac{\gamma t}{2}} & 0 \\
0 & \frac{2 \omega}{\beta} e^{-\frac{\gamma t}{2}}\sinh(\nicefrac{\beta t}{2}) & 0 & e^{-\frac{\gamma t}{2}}\left( \cosh(\nicefrac{\beta t}{2}) + \frac{\gamma}{\beta} \sinh(\nicefrac{\beta t}{2})\right)
\end{bmatrix}
\end{equation}
where $\beta = \sqrt{\gamma^2 - 4\omega^2}$.

If $\omega$ is large, and $\gamma$ is small (as we hope will be the case in low-noise systems), then $\beta$ will be imaginary, and the hyperbolic functions will become regular trigonometric functions:
\begin{flalign}
\beta &\equiv i \nu \\
\cosh \left(\frac{\beta t}{2} \right) & = \cos \left(\frac{\nu t}{2} \right)\\
\frac{\sinh(\frac{\beta t}{2})}{\beta} & = \frac{\sin(\frac{\nu t}{2})}{\nu} \\
S & \mapsto \begin{bmatrix}
1 & 0 & 0 & 0 \\
0 & e^{-\frac{\gamma t}{2}}\left( \cos(\nicefrac{\nu t}{2}) - \frac{\gamma}{\nu} \sin(\nicefrac{\nu t}{2})\right) & 0 & -\frac{2 \omega}{\nu} e^{-\frac{\gamma t}{2}}\sin(\nicefrac{\nu t}{2}) \\
0 & 0 & e^{-\frac{\gamma t}{2}} & 0 \\
0 & \frac{2 \omega}{\nu} e^{-\frac{\gamma t}{2}}\sin(\nicefrac{\nu t}{2}) & 0 & e^{-\frac{\gamma t}{2}}\left( \cos(\nicefrac{\nu t}{2}) + \frac{\gamma}{\nu} \sin(\nicefrac{\nu t}{2})\right)
\end{bmatrix}
\end{flalign}
We have control over $t$, and we'd like to determine how to set it in order to obtain the maximum-fidelity $Y_{90}$. 
To separate the noise from the gate we'd like to perform, we express the total superoperator as the product of a desired $Y_{90}$ superoperator and a noise operator:
\begin{flalign}
S &= S_{\textrm{Noise}}S_{\textrm{Id}}\\
\therefore S_{\textrm{Noise}} &= S_{\textrm{Id}}^{-1} S \\
S_{\textrm{Id}} &= \begin{bmatrix}
1 & 0 & 0 & 0 \\ 0 & 0 & 0 & 1 \\ 0 & 0 & 1 & 0 \\ 0 & -1 & 0 & 0 
\end{bmatrix} \\ 
\therefore S_{\textrm{Noise}} &= \begin{bmatrix}
1 & 0 & 0 & 0 \\
0 & \frac{2\omega}{\nu} \exp(-\frac{\gamma t}{2}) \sin(\frac{\nu t}{2}) & 0 & \exp(-\frac{\gamma t}{2}) \left( \frac{\gamma}{\nu} \sin(\frac{\nu t}{2}) -\cos (\frac{\nu t}{2})\right) \\
0 & 0 & \exp(-\gamma t) & 0 \\
0 & \exp(-\frac{\gamma t}{2}) \left( \frac{\gamma}{\nu} \sin(\frac{\nu t}{2}) + \cos (\frac{\nu t}{2})\right) & 0 & \frac{2\omega}{\nu} \exp(-\frac{\gamma t}{2}) \sin(\frac{\nu t}{2})
\end{bmatrix} 
\end{flalign}
To find the channel fidelity $F_{\Lambda} = \elem{\Omega}{\Lambda \otimes \id \left( \proj{\Omega} \right)}{\Omega}$ (where $\ket{\Omega}$ is a Bell state), we take the trace of this superoperator and divide by 4 (I won't prove this here, but leave it as an exercise):
\begin{equation}
F_{S_{\textrm{Noise}}} = \frac{1}{4} \left[ 1 + \frac{4\omega}{\nu} \exp\left(-\frac{\gamma t}{2}\right) \sin\left(\frac{\nu t}{2}\right) + \exp\left(-\gamma t\right) \right]
\end{equation}
To find out how long to leave the Hamiltonian on, we optimize this fidelity over $t$:
\section{Questions}
\begin{enumerate}
\item Show that ``shorting'' the gate time optimally (maximizing the channel fidelity) doesn't appreciably raise the fidelity over just setting $\omega t = \frac{\pi}{2}$.
\item How does all this change when we add amplitude damping? 
\end{enumerate}

\end{document}